\section{Background}

\label{sec:background}

\subsection{The Information Aggregation Problem}

Consider the following problem within algorithmic mechanism design known as the
\emph{information aggregation problem}~\cite[Ch.~26]{AGTBook}. An individual
known as the ``aggregator'' wishes to obtain a prediction about an uncertain
variable which will be realised at some point in the future. There are a number
of individuals known as the ``informants'' who each hold sets of information
about the variable's outcome. The goal is to design a mechanism that extracts
the relevant information from the informants and uses it to provide a
prediction of the variable's realisation. In an ideal setting the mechanism
should produce the same prediction as an omniscient forecast that has access to
all information available to all informants. This may not be viable in
practice, since each agent's information is private, and so the mechanism must
incentivise them to act in the desired truth-telling manner.

A prediction market is one mechanism that can provide such a forecast. In this
setting the aggregator creates a financial security whose payoff is tied to the
outcome of the variable. In the simpler case of binary events, such a security
may pay out \$1 for each share held if the variable has a ``true'' or ``yes''
outcome, and \$0 otherwise, however other types of markets can be created,
including discrete (``will the result of the match be a home win, away win, or
a draw?''), continuous (``what will be the highest measured temperature in
Coventry in September?''), or any combination of these types. Informants are
then able to participate in the market induced by the security by trading
shares according to their beliefs: those who believe, for example, that global
warming is real might buy shares at a given price in the market, ``the global
average temperature in 2020 will be higher than that in 2019'', while deniers
may be inclined to sell shares. The share price will be adjusted accordingly,
and the aggregator can view the current price as the informant's combined
belief of the outcome of the event.

In the \emph{partition model of knowledge} there is a set $\Omega$ of possible
states of the world and at any moment the world is in exactly one state $\omega
\in \Omega$, though the informants do not know which. Each informant $i$ may,
however, possess partial information regarding this true state, represented by
a partition $\pi_i$ of $\Omega$. The agent knows in which subset of this
partition the true world state lies, but does not know the exact member of
which is true. Given $n$ agents, their combined information $\hat{\pi}$ is the
coarsest common refinement of the partitions $\pi_1, \ldots, \pi_n$.\footnote{A
partition $\alpha$ of a set $X$ is a refinement of a partition $\rho$ of $X$ if
every element of $\alpha$ is a subset of some element of $\rho$. In this case
$\alpha$ is \emph{finer} than $\rho$ and $\rho$ is \emph{coarser} than
$\alpha$.}

We also assume a common prior probability distribution $P \in \Delta^{\Omega}$
which describes the probabilities that all agents assign to the different world
states before receiving any information. Once each agent receives their partial
information, they form their posterior beliefs by restricting the common prior
to the subset of their partition in which they know the true state to lie.  A
\emph{forecast} is an estimate of the expected value of the function $f :
\Omega \rightarrow \{0,1\}$, known as an \emph{event}, which equals one for
exactly one subset of $\Omega$ and 0 otherwise.

As mentioned, there is an ideal ``omniscient'' forecast that uses the
distribution $P$ restricted to the subset of $\hat{\pi}$ in which the true
world state lies -- this is difficult to achieve given the private nature of
each agent's information. The goal is therefore to design a mechanism to
incentivise agents to reveal their private information such that in equilibrium
we achieve a forecast as close as possible to the omniscient one. Prediction
markets offer the agents the chance of financial gain for revealing information
on the expected value of $f(\omega)$, and the share price of the security can
be interpreted as the collective forecast of the agents. In the remainder of
the section we shall outline some of the different approaches to designing the
market mechanism: the best choice will vary for the setting and will depend on
what is tractable given the types of bets that are on offer. First we shall
introduce combinatorial markets.

\subsection{Combinatorial Prediction Markets}

A \emph{combinatorial} prediction market is one in which the total state space
$\Omega$ is the product space of a collection of base events. Suppose knowing
the outcome of a given event cannot be predicted with certainty even if the
outcomes to all other base events are known, and we wish to provide the
opportunity to trade on any outcome $\omega \in \Omega$: with a set of base
events of size $\mathcal{|E|}$ we would have a total outcome space of size
$2^\mathcal{|E|}$.  With such a large outcome space it may therefore be
impossible to even all list the securities available in such a market.
Combinatorial markets can make use of ``expressive'' bidding languages that
represent collections of securities succinctly, including \emph{combined
orders} and \emph{compound orders}.

Combined orders allow a user to trade a collection of securities by specifying
the securities they wish to buy or sell together as a bundle along with limit
prices for each constituent security. If there is even a single security in the
bundle that is not at least as ``good''\footnote{If we are buying we want a
price less than or equal to the limit price, while the opposite is true if we
are selling.} as its limit price, then no trade is made. If an agent were to
trade on these securities in a non-combinatorial market, they would need a
trade to be executed on each one individually. Throughout the execution of such
a sequence of trades, however, market prices would be subject to change, and in
the worst case these fluctuations could reduce or even reverse the utility of
participating in such trades. Combined orders therefore protect its
participants from such risk. Calculating the assignment of securities to buyers
in such a setting is as hard as the winner-determination problem faced by a
combinatorial \emph{auction}, which known to be NP-hard.

Compound orders are generalisations of combined orders and allow users to trade
on any Boolean expression on the set of base events. Now again the size of the
outcome space is $2^\mathcal{|E|}$, but now there are $2^{2^\mathcal{|E|}}$
subsets of these outcomes expressible with Boolean formulae. Agents now place
orders by requesting $q$ shares of the security $S_{\phi|\psi}$ at share price
$p$, where $S_{\phi|\psi}$ pays out \$1 if both Boolean formulae $\phi$ and
$\psi$ are true, \$0 if only $\psi$ is true, and refunds the user if $\psi$ is
false. These orders will yield a payoff $\gamma^{\langle \omega \rangle}$
depending on the state $\omega \in \Omega$, and can be written as:
%
\[
	\gamma^{\langle \omega \rangle} = q \cdot \mathds{1}_{\omega \in \psi}
	(\mathds{1}_{\omega \in \phi} - p)
\]
%
in which $\omega \in \phi$ means outcome $\omega$ satisfies the Boolean formula
$\phi$. This says that the order will receive an overall payoff of zero if the
true world state $\omega$ does satisfy $\psi$, meaning the order is invalid and
refunded to the user. Otherwise, the user will receive a payout of $(1-p)$ for
each share they bought if $\omega$ satisfies $\phi$. If the event does not
occur they will lose what they paid and receive a payoff of $-p$ for each of
$q$ shares. Each user $i$ has a payoff vector $\gamma_i$ induced by submitting
orders. It is the job of the auctioneer to determine which orders to accept.
Let $\alpha_i \in \{0,1\}$ indicate whether the auctioneer accepts an order
from user $i$. Since they will collect the money paid by the trader and payout
their winnings according to each user's payoff vector, the auctioneer receives
payoff:
%
\[
	\gamma_a = \sum_i -\alpha_i \gamma_i
\]
%
The \emph{indivisible matching problem} asks that, given a set of orders, does
there exist a set of $\alpha_i \in \{0,1\}$ and $\sum_i \alpha_i \ge 1$ such
that for any outcome $\omega$ the auctioneer receives payoff $\gamma_a^{\langle
\omega \rangle} \ge 0$? In other words, the auctioneer is looking to accept
some bundle of orders \emph{without risk}.

\textbf{Example} Suppose $\mathcal{E} = \{X_1, X_2\}$ and there are two orders:
agent 1 wishes to buy two shares of security $S_{X_1}$ at \$0.6 per share while
agent 2 wishes to sell one share of security $S_{X_1,X_2}$ for \$0.2 per
share. Assuming the auctioneer accepts both orders, the payoffs are as follows
(note player 2's payoff vector is negated since they are selling):
%
\begin{equation*}
	\begin{aligned}
		\gamma_1 & = \left\langle \begin{matrix}
			\gamma_1^{X_1,X_2} &
			\gamma_1^{X_1,\bar{X_2}} &
			\gamma_1^{\bar{X_1},X_2} &
		\gamma_1^{\bar{X_1},\bar{X_2}} \end{matrix} \right\rangle 
%
		 = \left\langle \begin{matrix}
			2(1-0.6) & 2(1-0.6) & 2(-0.6) & 2(-0.6) \end{matrix} \right\rangle \\
%
		\gamma_2 & = - \left\langle \begin{matrix}
			1(1-0.2) & 1(-0.2) & 1(-0.2) & 1(-0.2) \end{matrix} \right\rangle \\
%
		\gamma_a & = \left\langle \begin{matrix}
			0 & -1 & 1 & 1 \end{matrix} \right\rangle
	\end{aligned}
\end{equation*}
%
Accepting both orders is therefore not a valid solution to the indivisible
matching problem since in the event that $\omega = \{X_1,\bar{X_2}\}$ the
auctioneer would have to run a loss. In fact, finding a solution to the
indivisible matching problem is NP-complete~\cite[Ch.~26]{AGTBook}. Although we
do not focus on combinatorial prediction markets in this project, it is useful
to illustrate the challenges of designing efficient market mechanisms. In
particular, we are concerned with low liquidity, which arises when there is too
little participation in the market for a trade to be executed.

\subsection{Trading Mechanisms}

\label{sec:tradingMechanisms}

\subsubsection{Continuous Double Auctions}

Trading mechanisms are responsible for deciding which trades are executed, in
what quantities, and at which price points. In the previous section, although
users had to submit orders of a particular form, the auctioneer was free to
accept any subset of the orders received: this section expounds upon \emph{how}
trades may be chosen to be executed. These are called market trading
mechanisms, and include double auctions, market call auctions, and automated
market makers. A Continuous Double Auction (CDA) is a mechanism in which buyers
are matched with sellers of a particular security. The market maker keeps an
order book that tracks the bids, submitted by those looking to buy, and the
asks, submitted by those looking to sell. Traders arrive asynchronously and
place orders, and when two opposite orders match the trade is executed. CDAs
are traditionally used in highly liquid markets, such as the New York Stock
Exchange, where there are many bids for a given ask and vice versa. An issue
with this approach is that it relies on there always being a willing buyer and
seller at a particular price and quantity in order to execute a trade.
Prediction markets have far fewer participants than stock exchanges, and the
problem of low liquidity is made even worse in combinatorial markets, in which
a trader's attention is split among an exponential number of securities. This
makes the likelihood that a buyer and seller are looking to trade on the same
event exceptionally small. Prices may also not be informative of the true
beliefs held by traders in CDAs: since all traders can see bids and asks as
they are submitted sellers may be encouraged to ask less than what they truly
believe to be a security's ``true'' value in order to undercut another seller
and make a profit. This is not useful in a prediction market setting, where we
want to elicit the true beliefs from users.

These problems arising from market low liquidity can be averted by using an
automated market maker, in which a price maker is nearly always willing to
accept both buy and sell orders at a certain price. Participation in the market
will have an effect on these prices, the exact nature of which will be down to
the market maker. This ensures that, if the price is desirable, participants
are always able to make a trade. An automated market maker is not typically
used in real-world markets since always assuming the opposite side to any trade
would likely result in significant losses for ``the house''; in play-money
markets this is not such an issue as losses are less detrimental and have no
real-world negative value. In general there are three properties that an
automated market maker should satisfy in order to be of practical use: traders
should have an incentive to participate in the market whenever their beliefs
would change the price; the computation of market prices should be tractable;
and the market's loss should be bounded. We present two options for
implementing automated market makers: as a parimutuel market or using a Market
Scoring Rule.

\subsubsection{Parimutuel Markets}

In parimutuel markets traders wager money their choice of outcomes from a
mutually exclusive and exhaustive set. When the event's outcome is realised the
total wagered money is split between those who wagered correctly, in proportion
to the size of their bet. In order to accommodate traders selling shares prior
to the outcome of the event, \emph{dynamic} parimutuel markets incorporate a
cost function that varies the price of a single share due to trading activity.
An example of one such cost function is the share-ratio cost function. Suppose
we have an outcome space $\Omega$ and let $q_j$ denote the total number of
shares for event $j \in \Omega$. Let $\vect{q} = (q_1, \ldots, q_{|\Omega|})$
be the vector of outstanding shares of all contracts. The share-ratio cost
function is:
%
\[
	C(\vect{q}) = \sqrt{\sum_j q_j^2}
\]
%
A trader wishing to buy $q_j' - q_j$ shares of security $j$, thus changing the
number of outstanding shares of $j$ from $q_j$ to $q_j'$, pays the market
$C(\vect{q}_{-j}, q_j') - C(\vect{q})$.\footnote{We use $\vect{q}_{-j}$ to
denote the vector $(q_1, \ldots, q_{j-1}, q_{j+1}, \ldots, q_{|\Omega|})$.}
There is a corresponding price function $p_j$ that gives the price for
purchasing an infinitesimal quantity of shares in security $j$ and is used to
quote a share price to market participants:
%
\[
	p_j = \frac{q_j}{\sum_k q_k^2}
\]
%
This share price $p_j$ should not be used to calculate the cost of a
transaction since an agent's participation in the market will instantly change
this value. The purpose of a prediction market is to elicit private information
on some future event: we use the information we have on user participation in
the market to compute probability $\pi_j$ of outcome $j$ as $\pi_j = p_j^2$.

\subsubsection{Scoring Rule Markets}

A scoring rule is used to assign probabilities to a set of mutually exclusive
outcomes. When the scoring rule is proper\footnote{A scoring rule giving the
highest expected reward for reporting the true distribution.}, it can be
converted to an automated market maker that uses a Market Scoring Rule
(MSR)~\cite{Hanson2003}.  Again at the heart of a MSR is the cost function
$C(\vect{q})$, which is a means of recording the total amount of money spent in
the market by traders as a function of the total number of shares in
circulation. Traders wishing to purchase $q_j' - q_j$ shares of security $j$
must again pay $C(\vect{q}_{-j}, q_j')-C(\vect{q})$. Note that in both
parimutuel markets and scoring rule markets, these rules also encode sell
transactions, in which case $q_j' < q_j$. We cannot use $C$ directly to quote a
share price to the user since we first need to know what quantity they to buy
or sell: we use its derivative $p = \partial C / \partial q_j$ to again quote
the cost for an infinitesimal quantity of shares.  As an example, the quadratic
scoring rule gives some reward $Q(\vect{r},i)$ if the $i$th event occurs, given
the probability vector $\vect{r}$. The cost function corresponding to the
quadratic scoring rule is:
%
\[
	C(\vect{q}) =
	\frac{\sum_j q_j}{|\Omega|} + \frac{\sum_j q_j^2}{4b}  +
	\frac{\left( \sum_j q_j \right)^2}{4b|\Omega|} - \frac{b}{|\Omega|}
\]
%
Scoring rule markets typically pay out \$1 for each share held of a security
whose outcome was positive, and \$0 otherwise. In contrast, parimutuel markets
can have a different payoff per share since they pay out an equal portion of
the total amount wagered per share held to winning shareholders. This project
implements the former. The following section will outline some existing
prediction markets as well as important results from the literature.
