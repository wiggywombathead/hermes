\section{Design}

% different types of market:
% - market scoring rule
% - call market auction
% - continuous double auction
% - bookmaker
% - parimutuel market

\label{sec:design}

\subsection{Tools}

The project is implemented in Common Lisp and the code has been developed and
tested within the Steel Bank Common Lisp (SBCL) compiler and runtime system.
Code version control has been achieved with Git and Github.  Writing the web
application has required the use of several libraries available from the
library manager Quicklisp, specifically:

\begin{itemize}
	\itemsep0em
	\item Hunchentoot
	\item CL-WHO
	\item Mito
	\item SXQL
	\item Parenscript
	\item Smackjack
\end{itemize}

Hunchentoot provides the environment on which we host the server. Most
importantly it provides automatic session handling, allowing for multiple users
to be logged in at once, and easy access of GET and POST parameters, enabling
interaction via HTML forms. To generate the webpages, we use CL-WHO, which
converts Lisp statements into strings of valid HTML. Defining webpages while
remaining in the Lisp environment means we may use Lisp's macro system to build
abstractions for both defining structure and processing data in one interface.

Mito and SXQL provide the ability to connect to and interact with a Relational
Database Management System (RDBMS). We use MySQL, though this choice is largely
immaterial given our simple requirements of the database.

Parenscript incorporates Javascript into the site with the goal of improving
user experience. Currently it is only used to ensure that all necessary fields
during market creation, trading, and market resolution are non-empty to avoid
sending incomplete data to the server. It will be used to a greater degree in
the future to ensure responsiveness: all information displayed to the user must
be current to ensure that users are interacting with an up-to-date state of the
system. We therefore plan to make more use of Parenscript and Smackjack, an
AJAX library for Lisp, to allow for asynchronous interaction with the server.
This will include, for example, continuously updating a stock's price or
calculating the cost of a transaction without a page refresh, and stronger
client-side validation.

\subsection{System Overview}

We implement the mechanism for peer prediction introduced by Freeman et
al~\cite{CODiPM}. In this section we will introduce the main ideas they present
and give an overview of the mechanism itself.

We are interested in setting up a prediction market for the outcome of the
binary event (random variable) $X \in \{0,1\}$. The terms ``market'' and
``security'' are used interchangeably to refer to the entity comprising a wager
(e.g. ``Arsenal will beat Tottenham'') and a deadline -- these two pieces of
information are all we need to represent event $X$. The mechanism is divided
into two main stages: the market stage, where users may buy and sell shares in
the securities whose deadlines have not yet passed; and the arbitration stage,
where a subset of the users report on the outcome of the security and we
compute the payout price per share held. A key feature of this mechanism is
that users may act as arbiters in markets in which they themselves hold shares.

Since we rely heavily on user participation for the mechanism to run correctly,
it is important that users act in the desired manner. This mechanism
incentivises truthful reporting in the arbitration stage, whereby it is in a
user's best interests to report what they truly believe a market's outcome to
be and not attempt to manipulate the system, even if the user is a stakeholder
in the market. It also allows us to achieve certain guarantees, such as a bound
on the amount we must pay to arbiters to reward them for participation, which
serves as a guide when setting system parameters.

\subsubsection{Market Stage}

The market stage allows users to create markets for any event they wish and
trade shares in these markets. Since we crowdsource outcome determination,
there is no restriction placed on the types of bets users may make other than
that their result is a ``yes'' or a ``no''. Ambiguous bets are allowed, though
likely to suffer in the arbitration stage, since users may have different
interpretations on the outcome.

In order to create a market for a user-specified event, we use a \emph{market
scoring rule (MSR)}, which is a means of assigning a probability to a set of
mutually outcomes. After the wager and deadline have been specified, the MSR
simply takes into account the number of shares bought and sold and returns a
probability $p \in [0,1]$. This describes the perceived likelihood amongst
users of the event having a positive outcome. We can also use this to set the
share price of the security.  Using a MSR is different from a traditional
market in that there is not a fixed number of shares in circulation: instead,
buying into a market increases the total number of shares and selling decreases
it. Let $q$ denote the total number of shares of a given security. An agent
wishing to buy $q'-q$ shares (i.e. increasing the total number of shares to
$q'$) will pay $C(q')-C(q)$, for our choice of convex, differentiable,
monotonically increasing scoring rule $C$. We may tune the behaviour of this
cost function by using a liquidity parameter $b$, such that $C_b(q) := b \cdot
C(q/b)$, which controls the responsiveness of $C$. A lower value of $b$ means
that the share price changes more quickly for fewer shares bought, and vice
versa.

The market stage implements trading fees in order to raise the funds necessary
to pay stakeholders when the event's outcome is realised, and to reward users
for participation in the arbitration stage. Buying shares pushes the share
price $p$ towards \$1, while selling shares pushes it towards \$0, and the fee
can be interpreted as a fee on the worst-case loss that an agent incurs. For a
fixed parameter $f$, a ``buy'' transaction that pushes the share price to $p$
incurs an additional charge of $fp$, while a ``sell'' transaction incurs a
charge of $f(1-p)$. Transactions are only charged a fee when the user increases
their risk: if they are simply liquidating their position (selling shares they
own or buying back shares they have sold) then no fee is charged.  Users may
trade in the market as long as they have enough funds to make the transaction
and the deadline for the event has not passed. After the deadline has expired
the users' positions are final and we move on to determining the outcome of the
event in the arbitration stage.

\subsubsection{Arbitration Stage}

The arbitration stage is concerned with determining the perceived outcome of
event $X$ by taking reports from a subset of the users in the system, known as
the arbiters. Each arbiter $i$ receives a (private) signal $x_i \in \{0,1\}$
that tells them the result of the event -- this is analogous to reading the
news, watching the match, or even hearing about it from a friend. They then
submit a report $\hat{x}_i \in \{0,1\}$ to the system that tells us what they
believe the outcome to be. Note that since the signal they receive is private,
we have no way to know whether the user is reporting what they truly believe,
or whether they are trying to manipulate the system for their own gain.
Arbiters are then paired randomly, and paired arbiters $i$ and $j$ are paid a
reward of $u(\hat{x}_i, \hat{x}_j)$ according to the ``1/prior with midpoint''
mechanism, which we will detail below.

We can now determine the outcome of the market. In a traditional prediction
market, users are paid \$1 for each share owned (or they must buy shares back
at \$1 if they have shorted the security) if the event has a positive outcome,
and \$0 otherwise. Hence if an event is almost certain to occur, more users
will buy shares than sell them, as they expect they will be paid out for
holding shares in the market when the outcome is realised. This will push the
share price towards \$1. Similarly, if an event is believed to be unlikely,
more users will sell shares than buy them and the price will approach \$0. This
prevents arbitrarily large profits being made for risk-free bets. In our market
the outcome of an event is the random variable $\hat{X} \in [0,1]$ which is set
to the proportion of arbiters that submit a report of $\hat{x}_i=1$.
Stakeholders are then paid out in the usual way.

\subsubsection{1/prior mechanism}

\label{sec:oneOverPrior}

We use the 1/prior payment mechanism to reward users for submitting reports on
the outcomes of events, with a modification to incentivise truthful reporting.
The 1/prior payment mechanism was conceived by Jurca and
Faltings~\cite{JurcaFaltings2008, JurcaFaltings2011} as a means of rewarding
arbiters for participation in opinion polls, and Witkowski~\cite{Witkowski2014}
generalises this mechanism to pay out different amounts depending on the
signals reported by paired arbiters. For paired arbiters $i$ and $j$ with
reports $\hat{x}_i$ and $\hat{x}_j$, the 1/prior mechanism pays $i$ and $j$ as
a reward for reporting on the market outcome:

\begin{equation}
	\label{eq:oneOverPrior}
	u(\hat{x}_i, \hat{x}_j) =
	\begin{cases}
		k \mu & \text{if } \hat{x}_i = \hat{x}_j = 0 \\
		k (1-\mu) & \text{if } \hat{x}_i = \hat{x}_j = 1 \\
		0 & \text{otherwise}
	\end{cases}
\end{equation}

in which $k$ is a parameter and $\mu$ is the (common) prior belief that $X=1$.
A suitable value to use for $\mu$ would be the closing price for the market: if
users feel an event is likely to happen the share price will be pushed towards
\$1, while if they feel it is unlikely it will be pushed towards \$0.

The modification to the 1/prior mechanism is simple though requires two extra
values to be computed. Let $\mu_1^i$ be the probability that, given that agent
$i$ receives a positive signal, another randomly chosen agent also receives a
positive signal, and let $\mu_0^i$ be the probability that, given that agent
$i$ receives a negative signal, another randomly chosen agent receives a
positive signal. We require the ``update'' probabilities $\mu_1, \mu_0$ to be
common across all agents, which we can achieve by simply taking:

\begin{equation}
	\begin{gathered}
		\mu_1 := \min_i \mu_1^i \\
		\mu_0 := \max_i \mu_0^i
	\end{gathered}
\end{equation}

The modified payment method we use, the ``1/prior with midpoint'' mechanism, is
now simply Equation~\ref{eq:oneOverPrior} with $\mu$ replaced by $(\mu_1 +
\mu_0)/2$. This guarantees the incentives for arbiters are the same regardless
of their signal.


