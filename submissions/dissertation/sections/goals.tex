\section{Goals}

\label{sec:goals}

\subsection{Core Features}

The goal of this project is to implement a truthful decentralised prediction
market in which users may specify the bets on which to trade. The market
outcomes will be decided by peer prediction, in which events are settled by a
subset of the users known as arbiters. A user may even act as an arbiter in a
market in which they themselves may hold a stake. Specifically, we seek to:

\begin{itemize}
	\item create a web application on which users can create custom bets and
		trade on these markets
	\item implement a trading mechanism that allows users to buy and sell
		shares in the user-made securities using play money
	\item crowdsource outcome determination using reports from arbiters who may
		hold positions in the market
	\item incentivise truthful behaviour at all stages in the mechanism
\end{itemize}

These goals will largely be achieved by implementing the mechanism outlined by
Freeman et al.~\cite{Freeman2017}, albeit wit several practical modifications.
The first three goals cover core functionality of any decentralised prediction
market, while the fourth is concerned with the tuning of system parameters,
before and during execution, in order to ensure that users do not manipulate
the mechanism. In this users are not only discouraged from attempting to
``game'' the system for personal gain, they are hurt for doing so.

Other non-essential features but highly desirable for strong user experience
include asynchronous communication with the server in order to display
up-to-date pricing information to the user without a page refresh, and the
automated closing of markets. Both of these features would make the system
straightforward and intuitive to use. Moreover, it would allow the system to
run independently, meaning the market's functioning is only influenced by the
community, one of the key points of implementing a decentralised market.

As we will discuss in more detail in Section~\ref{sec:design}, one aspect of
the mechanism is the assumption that the system knows the signal error rates
when user's receive news about a market's outcome. This is unrealistic in
practice, since we cannot hope to know the exact where each user learns about
the outcome of the event nor the accuracy of said source's reporting. Hence we
also look to implement a way in which users do not need to be explicitly asked
their estimates of signal accuracy, and instead this is calculated based on
their past reporting history. This leaves less opportunity to game the system,
the entire point of implementing this prediction market mechanism.

\subsection{Stretch Features}

With more time, there are plenty of additional features that could be
implemented to render the system more intuitive and usable. These include the
option to create different types of markets, particularly categorical ones
since they would function similarly to binary markets but allow multiple
related markets to be expressed more succinctly. Furthermore the option to
create and sort markets by categories would help users offer their information
more readily if they are especially interested in a certain topic, say politics
or sport.

A useful feature to implement would be the tracking of price histories for each
security. This would enable graphs to be generated so that users could be more
informed on how the forecast of an event has changed over time. This would make
the decision to participate at a particular price point more interesting on
their part since they are not only estimating their belief on the probability
that the event will have a positive outcome, but also on how the other users
will act, similar to a real stock market.

Finally, an issue with the mechanism of Freeman et al. as it stands is that it
does not directly punish users for creating markets on ambiguous bets. Traders
may become confused about the wording, leading to different interpretations and
hence different users trading on different beliefs -- this is not useful for
information aggregation. Although the negative effects are somewhat mitigated
in that the very same community that trades in the security also decides on its
outcome, there is no mechanism in place to specifically encourage clear bets.
This is something that could be improved and would be useful in avoiding the
market becoming swamped with overly subjective wagers.

\subsection{Motivation}

As discussed, there are already numerous prediction markets that exist in the
literature. The Iowa Electronic Markets are the longest running and arguably
most successful, but the options offered to the users on which to trade are too
restrictive. This is a similar issue among all centralised markets, including
InTrade, PredictIt, and the Hollywood Stock Exchange. While their success can
be attributed to their narrow focus on a particular topic, it seems more
interesting to be able to aggregate information from a wider variety of themes,
sacrificing perhaps some of the predictive accuracy for more widespread
forecasts to be made.

Decentralised prediction markets are not a new concept, however current
examples in the literature lack in the functionality they offer. In the case of
\emph{Omen}, while they allow any user to create a market, they rely on a
single oracle to determine the outcome of the event. This leaves a single point
of ``failure'' in the system and leaves it open to being manipulated. For
example, listing a biased news source as the oracle could have a significant
effect on the event's outcome. Although this is mitigated somewhat by
displaying to traders the oracle chosen, this still encourages them to trade on
how they believe the oracle will report the market and not necessarily the
market itself. The mechanism by which \emph{Augur} determines market outcomes
appears to be an improvement over this, in which multiple reporters from the
community back their report of the market outcome with \$REP tokens, thus
implementing a form of reputation system. However, it does not deal with
ambiguous bets elegantly, offering the option for a report a market's
outcome as ``invalid''. Given the inevitability of such markets in a
decentralised setting, this is a key weakness to \emph{Augur}.

It is therefore well justified to implement the decentralised peer prediction
mechanism of Freeman et al. This not only allows users to create markets for
any event they see fit, but also crowdsources market outcomes by relying on
reports from the community. Thus instead of relying on a single source of
information, which leaves it vulnerable to reporting biases particularly for
more ambiguous or subjective bets, it gets a more complete picture of how the
users themselves, who are the ones interacting with the market, observed the
outcome. This can average out the biases present in any one news source. This
also seems to be a better method to deal with ambiguity than in \emph{Augur},
since the outcome of the market can be influenced by a reporter's
interpretation of a wager. Another issue with \emph{Augur} is that it has not
been shown to achieve any theoretical guarantees, which should be a key
consideration in an environment in which rational selfish agents are
interacting. Instead, the market we implement is proven to be incentive
compatible, meaning it it in a user's best interests to report market outcomes
truthfully and not attempt to manipulate it for their own gain. Although the
mechanism is not budget balanced, it can be fully subsidised by a trading fee
on each transaction, further making it practical and self-sufficient.
