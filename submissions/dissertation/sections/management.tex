\section{Project Management}

\label{sec:projectManagement}

\subsection{Methodology}

The project has been developed incrementally, with a focus on integrating new
functionality completely before progressing to new features. This approach is
well-suited to this project's design: since it comprises of mainly four
separate areas which are drawn together at the end, it is possible to focus on
implementing a feature within one area without it affecting the rest. As a
result, testing has been performed throughout and ensures that a newer version
of the project is never worse than its predecessor. Using Git and Github has
been helpful in this regard, providing cloud storage and the ability to roll
back to previous versions if the current one is broken by a new feature.

\subsection{Ethics}

There is little ethical consideration required for the development of this
project. All development and testing has been done independently, and all
resources used to implement the system is available freely. Testing has been
performed externally only to small extent, and even then only informally
through asking of colleagues' opinions. There is one ethical issue that faces
prediction markets in general, however, and the one we implement is no
exception. When prediction markets operate, directly or indirectly, with real
money, they run the risk of devolving into ``assassination
markets''~\cite{assassinationMarkets, crowdfundingMurder}. This refers to the
incentive that people may have to act in a way that changes the ``natural''
outcome of an event. In the extreme case, a bet may be made on the death of a
high-profile individual and an assassin could stand to make a large by
participating in the market and ensuring that the individual dies on a given
date. A way to sidestep this issue is to use virtual money with no real-world
value, and this is the approach we take.

\subsection{Scheduling}

There have been few issues regarding the scheduling of the project, though it
has benefited from a slight rearrangement of the initial timetable.
Figure~\ref{fig:old-schedule} shows the schedule as it was planned at the time
of the presentation, towards the start of the project's initial development.
Tasks have also taken slightly longer to implement, and a revised version is
given in Figure~\ref{fig:new-schedule}, which details both how time on the
project was spent and how we now plan to use the remaining time.

\begin{figure}[h]
	\centering
	\includegraphics[width=\textwidth]{old-schedule}
	\caption{Project timetable as it was initially planned}
	\label{fig:old-schedule}
\end{figure}

Firstly, although an early demonstration of the market and arbitration stages
was complete for the presentation, this did not yet use any persistent storage
or actual users. Therefore, to get these stages working as they would in the
final project, the website and database interface first needed implementing.
This has taken the majority of the time since the presentation, and includes
defining the various pages and forms with which the users interact with the
markets, as well as how we interact with the database to register, log in, and
pay users. As we will cover in the next section, there are only three main
areas of functionality left to target: asynchronous communication with the
server via AJAX calls, automation of the system regarding closing markets once
enough arbiters have submitted reports, and most importantly parameter tuning.
The latter has been given a large block of time to complete since there are
many values on which certain parameters depend, and it is imagined that
calculating these accurately will require care.

Secondly, the time dedicated to writing the interim report has been extended by
two weeks to reflect the extension to its deadline. Although it was not
initially required, the extra time that could be diverted towards developing
the systems while the writing of the report was delayed was useful in providing
more material to discuss. This was also caused in part by more focus than
anticipated being paid towards exam preparation.

\subsection{Next Steps}

% Parameter tuning which requires processing of all data submitted 
% Asynchronous database queries to keep page details accurate

The next task is to tune the parameters of the system to ensure truthful
reporting, since this is the entire reason for implementing the mechanism. This
includes setting transaction fee $f$ to the appropriate percentage and the
payment parameter $k$. It will also be important to fully automate the system,
meaning the server automatically decides when to close markets: this is
currently done by prompting the system to compute market outcomes, which is
insufficient. The final important feature to add is asynchronous server
communication -- although it does not necessarily add new functionality to the
core of the prediction market, it would dramatically increase usability. 

We plan to dedicate a significant portion of August to the writing of the
dissertation, to spread the workload more evenly and to allow features that are
developed later more time to be added into the report.  We hope to get feedback
on a draft of the dissertation three weeks before the final deadline so that
the following weeks can be spent refining the material, and for more versions
to be produced in the remaining time. This means we intend to get all required
goals working well before the end of the month, to ensure we are not making
considerable changes shortly before the deadline.

\begin{figure}[h]
	\centering
	\includegraphics[width=\textwidth]{new-schedule}
	\caption{Revised project timetable}
	\label{fig:new-schedule}
\end{figure}


