% Modelo de slides para projetos de disciplinas do Abel
\documentclass[9pt]{beamer}

\usetheme[progressbar=frametitle]{metropolis}
\usepackage{appendixnumberbeamer}
\usepackage[numbers,sort&compress]{natbib}
\bibliographystyle{plainnat}

\usepackage{booktabs}
\usepackage[scale=2]{ccicons}

\usepackage{xspace}
\newcommand{\themename}{\textbf{\textsc{metropolis}}\xspace}

\usefonttheme[onlymath]{serif}

\usepackage{listings}

\lstdefinestyle{mystyle}{
    basicstyle=\ttfamily\footnotesize,
    showlines=false,
    breakatwhitespace=false,         
    breaklines=false,                 
    captionpos=b,                    
    keepspaces=false,                 
    numbers=left,                    
    numbersep=5pt,                  
    showspaces=false,              
    showstringspaces=false,
}

\lstset{style=mystyle}

\title{A Decentralised Prediction Market}
\subtitle{CS907 Presentation}
\date{\today}

\author{Thomas Archbold}
\institute{Department of Computer Science, University of Warwick}
\titlegraphic{\hfill\includegraphics[height=1.5cm]{warwick-logo.png}}

\begin{document}

\maketitle

%\begin{frame}{Table of contents}
%  \setbeamertemplate{section in toc}[sections numbered]
%  \tableofcontents[hideallsubsections]
%\end{frame}

% introduction to prediction markets
% introduction to my project
% existing literature
% design overview
% current progress
% next steps
% demo
% questions

\section{Introduction}

\begin{frame}[fragile]{Prediction Markets}
    \onslide<1->{\textbf{Prediction Markets} allow you to trade on the outcomes of future events} \\~\\
    
    \onslide<2->{Market prices indicate how its traders feel about the outcome of an event}
    \begin{itemize}
        \item<3-> traders have different beliefs/knowledge informing their decisions
        \item<4-> provides means of aggregating information on topics of interest via crowdsourcing (``wisdom of the crowd'')
    \end{itemize}~\\
    
    \onslide<5->{\textbf{Securities} are traded between 0 and 100\%, and the payoff is contingent upon the outcome}
    \begin{itemize}
        \item<6-> typically, one share pays out \$1 in the event the outcome occurs, otherwise \$0
    \end{itemize}
\end{frame}

\begin{frame}{Centralised Prediction Markets}
    \onslide<1->{Traditional prediction markets are \textbf{centralised}
    \begin{itemize}
        \item trusted centre creates and closes market, manages transactions
        \item determines the outcome of the event and pays out accordingly
    \end{itemize}}~\\
    
    \onslide<2->{This restricts the types of bets that can be made
    \begin{itemize}
        \item must be explicitly offered by market maker/system
        \item high workload on trusted centre
    \end{itemize}}
\end{frame}

\begin{frame}{Decentralised markets}
    \onslide<1->{A \textbf{decentralised market} allows anyone to create markets for events
    \begin{itemize}
        \item event outcome is determined by consensus among users (\emph{arbiters})
    \end{itemize}}~\\
    
    \onslide<2->{Removes the need for a trusted centre \textsc{but}:}
    \begin{itemize}
        \item<3-> bets may become ambiguous, since there is no central moderator
        \item<4-> arbiters may manipulate the outcome of the market
        \item<5-> users still need to bet based on their true beliefs
    \end{itemize}~\\
    
    \onslide<6->{Freeman, Lahaie, and Pennock~\cite{CODiPM} introduced a mechanism that tackles these issues}
\end{frame}

\begin{frame}{Project Aims}
    \onslide<1->{\textbf{Goal:} implement a decentralised prediction market} \\~\\
    
    \onslide<2->{In particular:}
    \begin{itemize}
        \item<3-> create a web-application where users may make and trade bets with virtual money
        \item<4-> place no restrictions on the bets that can be made
        \item<5-> outcome is determined by a group of arbiters (who may hold a stake in the market)
    \end{itemize}~\\
    
    \onslide<6->{\textbf{Why is this worthwhile?}
    \begin{itemize}
        \item less restrictive way to crowdsource public opinion
        \item possibility for manipulation requires game-theoretic approach
    \end{itemize}}
\end{frame}

\begin{frame}{Related Work}
    \onslide<1->{Decentralised prediction markets based on cryptocurrencies:}
    \begin{itemize}
        \item<2-> \emph{Augur}
        \begin{itemize}
            \item reptutation system to protect against manipulation
        \end{itemize}
        \item<3-> \emph{Gnosis}
        \begin{itemize}
            \item a distributed exchange for ``tokens''
            \item uses ``trading rings'' to trade the large numbers of illiquid bets
        \end{itemize}
        \item<4-> \emph{Hivemind}
    \end{itemize}~\\
    
    \onslide<5->{\emph{Predictalot} was a centralised \textbf{combinatorial} prediction market
    \begin{itemize}
        \item combinatorial bids
        \item combinatorial outcomes
    \end{itemize}}
\end{frame}

\section{Design Overview}

\begin{frame}{Mechanism}
    \textbf{Market stage}
    \begin{itemize}
        \item set up market for event $X$ using a \textbf{market scoring rule} (MSR)
        \item agents trade in the market
        \item the market closes and trading stops
    \end{itemize}~\\
    
    \textbf{Arbitration stage}
    \begin{itemize}
        \item each arbiter receives signal (e.g. reading the news) and reports the outcome
        \item arbiters are paired randomly and paid according to 1/prior mechanism
        \item outcome of the market is the fraction of arbiters reporting $X=1$
    \end{itemize}
\end{frame}

\begin{frame}{Tools}
    Language: Lisp (SBCL) \\~\\
    
    Packages:
    \begin{itemize}
        \item Hunchentoot
        \item CL-WHO
        \item Parenscript
        \item Mito
    \end{itemize}
\end{frame}

\begin{frame}{Current Progress}
    \begin{itemize}
        \item Environment setup
        \item Skeleton code for mechanism
        \item Logarithmic Market Scoring Rule
        \item 1/prior payment mechanism
    \end{itemize}
\end{frame}

\begin{frame}[fragile]{Web server}
    The web server is run on Hunchentoot
    \begin{lstlisting}[language=Lisp]
    (setf *web-server* (make-instance 'hunchentoot:easy-acceptor 8080))
    (hunchentoot:start *web-server*)
    \end{lstlisting}
    
    HTML is generated using macros supplied by CL-WHO
    \begin{lstlisting}
    (cl-who:with-html-output-to-string ...)
    \end{lstlisting}
    
    Macros are used define page templates
    \begin{lstlisting}
    (defmacro define-url-fn ((name) &body body)
        ...)
    \end{lstlisting}
\end{frame}

\begin{frame}{Market stage implementation -- Continuous Double Auctions}
    \onslide<1->{We have two options to implement a prediction market:}
    \begin{itemize}
        \item<2-> Continuous Double Auction (CDA)
        \item<3-> Market Scoring Rule (MSR)
    \end{itemize}~\\
    
    \onslide<4->{In CDAs, an order book tracks bids and asks
    \begin{itemize}
        \item if a bid matches an ask, the trade is executed
        \item used in traditional stock markets
    \end{itemize}}
\end{frame}

\begin{frame}{CDA issues}
    \onslide<1->{\textbf{Problem:} low liquidity
    \begin{itemize}
        \item with few participants, there is no guarantee any bid will match an ask
    \end{itemize}}~\\
    
    \onslide<2->{\textbf{Solution:} an \emph{automated market maker} which always assumes the opposite side of a trade
    \begin{itemize}
        \item participants are always able to trade
        \item \textsc{BUT} the ``house'' is likely to lose money
    \end{itemize}}
\end{frame}
    
\begin{frame}{Market mechanisms -- Market Scoring Rules}
    \textbf{Scoring rules} measure the accuracy of probabilistic predictions
    \begin{itemize}
        \item we use the \textbf{Logarithmic Market Scoring Rule}~\cite{LMSR}
    \end{itemize}~\\
    
    \begin{equation}
        C_b (q) = b \cdot \log (1+e^{q/b})
    \end{equation}~\\
    
    where:
    \begin{itemize}
        \item $q$ is the quantity (number of shares)
        \item  $b$ is an arbitrary constant (``liquidity'' parameter)
        \begin{itemize}
            \item smaller value $\Rightarrow$ higher price responsiveness
        \end{itemize}
    \end{itemize}
\end{frame}

\begin{frame}{Logarithmic Market Scoring Rule}
    \onslide<1->{An agent wishing to buy $q'-q$ shares of a security pays $C_b(q')-C_b(q)$
    \begin{itemize}
        \item suppose $b=10$ and there are 10 outstanding shares
        \item Alice wishes to buy 7 shares $\Rightarrow$ $q'=17$
        \item she pays $C_{10}(17)-C_{10}(10)=\$5.54$
    \end{itemize}}~\\
    
    \onslide<2->{Share price is calculated with:
    \begin{equation}
        p(q) = \frac{e^{q/b}}{1+e^{q/b}}
    \end{equation}}~\\
    
    \onslide<3->{$p(q)$ is not used to charge users, only to calculate closing price of the market}
\end{frame}

\begin{frame}{Market stage -- event outcomes}
    \onslide<1->{The outcome of an event is not binary -- rather, it takes value $\hat{X} \in [0,1]$
    \begin{itemize}
        \item corresponds to the proportion that reported that the event occurred
    \end{itemize}}~\\
    
    \onslide<2->{Two nice properties:}
    \begin{enumerate}
        \item<3-> each question has well-defined, unambiguous outcome
        \item<4-> limits the influence a single arbiter can have on the value of a security
    \end{enumerate}
\end{frame}

\begin{frame}{Market stage -- trading fees}
    \onslide<1->{Trading fees imposed on the worst--case loss incurred by an agent
    \begin{enumerate}
        \item raises funds to pay the arbiters
        \item bounds security prices away from 0 and 1
    \end{enumerate}}~\\
    
    \onslide<2->{For \emph{liquidation} transactions, no fees are imposed
    \begin{itemize}
        \item users are selling shares they own, or buying back those they have sold
    \end{itemize}}~\\
    
    \onslide<3->{Otherwise, the user is charged
    \begin{itemize}
        \item $f p$ for a security bought at price $p$
        \item $f (1-p)$ for security sold at price $p$
    \end{itemize}}
\end{frame}

\begin{frame}{Arbitration stage -- 1/prior mechanism}
    Arbiters need to be incentivised to act truthfully, since they may hold a stake in the market \\~\\
    
    This is achieved using (a modified) 1/prior mechanism:
    \begin{equation}
        u(\hat{x}_i, \hat{x}_j) = \begin{cases}
        k \mu & \text{if } \hat{x}_i = \hat{x}_j = 1 \\
        k(1 - \mu) & \text{if } \hat{x}_i = \hat{x}_j = 0 \\
        0 & \text{otherwise}
        \end{cases}
    \end{equation}
    
    where:
    \begin{itemize}
        \item $k$ is an arbitrary constant
        \item $\mu$ is the prior probability that an agent receives a positive signal
    \end{itemize}
\end{frame}

\begin{frame}{Parameter tuning}
    The mechanism uses the following parameters:
    \begin{itemize}
        \item $b$ -- liquidity
        \item $k$ -- to pay the arbiters
        \item $f$ -- trading fee
        \item $\delta$ -- ``update strength'', measures how strongly correlated signals are between arbiters
    \end{itemize}~\\
    
    Once the system is running we will need to tune these to suit the market:
    \begin{itemize}
        \item number of participants, ambiguity of question, maximum budget of a user, etc.
    \end{itemize}
\end{frame}

\section{Project Management}

\begin{frame}{Change of focus}
    \onslide<1->{Project has changed slightly from that of the research proposal
    \begin{itemize}
        \item still looking to incorporate some aspects of it
    \end{itemize}}~\\
    
    \onslide<2->{Much of the coming term will be spent revising for exams
    \begin{itemize}
        \item basic functionality not far off working, but needs to work in a web application
        \item bulk of the project to be done over the summer
    \end{itemize}}
\end{frame}

\begin{frame}{Tasks}
    Once the skeleton code is finished it will need integrating into a \emph{website} \\~\\
    
    The next major tasks will be to:
    \begin{itemize}
        \item design and implement the database to store information on all users and markets
        \item creating the pages through which users will interact with the system
        \item tuning the parameters of the mechanism 
    \end{itemize}~\\
    
    Beyond this is the (stretch) goal of computing relationships between markets
    \begin{itemize}
        \item how does a bet placed on one security affect the odds of another?
    \end{itemize}
\end{frame}

\begin{frame}{Timetable -- out with the old...}
    \begin{figure}
        \centering
        \includegraphics[width=\textwidth]{old-timetable}
        \caption{Timetable as of the research proposal submission}
        \label{fig:oldTimetable}
    \end{figure}
\end{frame}

\begin{frame}{... in with the new}
    \begin{figure}
        \centering
        \includegraphics[width=\textwidth]{new-timetable}
        \caption{Revised timetable}
        \label{fig:newTimetable}
    \end{figure}
\end{frame}

\section{Demonstration}

\section{Thank you for listening!}
\framesubtitle{Any questions?}

\begin{frame}{References}
    \bibliography{references}
    \bibliographystyle{abbrv}
\end{frame}

\end{document}
